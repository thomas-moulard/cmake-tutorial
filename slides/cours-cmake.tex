% Copyright (C) 2012 by Thomas Moulard.
%
% Permission is granted to copy, distribute and/or modify this
% document under the terms of the GNU Free Documentation License,
% Version 1.3 or any later version published by the Free Software
% Foundation; with no Invariant Sections, no Front-Cover Texts and
% no Back-Cover Texts.  A copy of the license is included in the
% section entitled "GNU Free Documentation License".

\documentclass[hyperref={pdfpagelabels=false}]{beamer}

%%%%%%%%%%%%%%%%%%%%%%%%%%%%%%%%%%%%%%%%%%%%%%%%%%%%%%%%%%%%%%%%%%%%%%%%%%%%%%%%
%% LaTeX and Beamer setup.                                                    %%
%%%%%%%%%%%%%%%%%%%%%%%%%%%%%%%%%%%%%%%%%%%%%%%%%%%%%%%%%%%%%%%%%%%%%%%%%%%%%%%%

\usepackage{lmodern}

\usepackage{beamerthemesplit}
\usepackage{multimedia}

\usepackage{amsmath}
\usepackage{amsfonts}
\usepackage{array}

\usepackage{url}
\usepackage{multicol}

\usepackage{tikz, pgfplots}

%%%%%%%%%%%%%%%%%%%%%%%%%%%%%%%%%%%%%%%%%%%%%%%%%%%%%%%%%%%%%%%%%%%%%%%%%%%%%%%%
%% Custom macros.                                                             %%
%%%%%%%%%%%%%%%%%%%%%%%%%%%%%%%%%%%%%%%%%%%%%%%%%%%%%%%%%%%%%%%%%%%%%%%%%%%%%%%%
\newenvironment{changemargin}[2]{%
\begin{list}{}{%
\setlength{\topsep}{0pt}%
\setlength{\leftmargin}{#1}%
\setlength{\rightmargin}{#2}%
\setlength{\listparindent}{\parindent}%
\setlength{\itemindent}{\parindent}%
\setlength{\parsep}{\parskip}%
}%
\item[]}{\end{list}}

\newcommand{\putat}[3]{\begin{picture}(0,0)(0,0)\put(#1,#2){#3}\end{picture}}

% Use tuned Warsaw theme.
\mode<presentation>
{
  \usetheme{Madrid}
  \usecolortheme{whale}
  \setbeamercolor{uppercol}{fg=white,bg=blue!40}
  \setbeamercolor{lowercol}{fg=black,bg=blue!15}
  \beamertemplateshadingbackground{blue!5}{blue!20}
  \beamertemplatenavigationsymbolsempty
}


%%%%%%%%%%%%%%%%%%%%%%%%%%%%%%%%%%%%%%%%%%%%%%%%%%%%%%%%%%%%%%%%%%%%%%%%%%%%%%%%
%% Meta-information.                                                          %%
%%%%%%%%%%%%%%%%%%%%%%%%%%%%%%%%%%%%%%%%%%%%%%%%%%%%%%%%%%%%%%%%%%%%%%%%%%%%%%%%
\title{CMake tutorial}
\author[T. Moulard]{Thomas Moulard}
\date[January 2012]{LAAS robotics courses, January 2012}


%%%%%%%%%%%%%%%%%%%%%%%%%%%%%%%%%%%%%%%%%%%%%%%%%%%%%%%%%%%%%%%%%%%%%%%%%%%%%%%%
%% Document.                                                                  %%
%%%%%%%%%%%%%%%%%%%%%%%%%%%%%%%%%%%%%%%%%%%%%%%%%%%%%%%%%%%%%%%%%%%%%%%%%%%%%%%%
\begin{document}

\maketitle

\section{Why are we here?}

\begin{frame}
\frametitle{Why are we here?}

\begin{itemize}
  \item All of you will have to compile software using CMake...
  \item and most of you will have to write software using CMake!
\end{itemize}

\vspace{1cm}

CMake is used by:
\begin{itemize}
  \item KDE4, MySQL, VXL, YARP, ROS\ldots
  \item Various scientific packages (especially vision)
\end{itemize}
\end{frame}


\begin{frame}
\frametitle{Why should I use CMake?}

\ldots because compiling a software is a complex, tedious process
requiring various tools to work together!

\vspace{1cm}

CMake will help you compiling the piece software you need, however it
is not magic and understanding (a bit!) what happens is
\textbf{essential}.

\end{frame}


\section{The fastest compilation course on Earth}

\begin{frame}
\frametitle{The fastest compilation course on Earth (1)}

The goal: producing a binary file called \textbf{program}.

\vspace{0.3cm}

These files can be encoded using different format, the most common one
on Linux is called \textbf{ELF} (Executable and Linking Format).

\vspace{0.3cm}

This \textbf{program} contains \textbf{symbols} (i.e. subroutines).
By conditionally jumping toward these \textbf{symbols}, a program can
incorporate logic behaviors / logic statements (i.e. if, while, for).

\vspace{0.3cm}

The \textbf{symbols} contained in a \textbf{program} can be displayed
using \texttt{objdump} or \texttt{nm}.

\end{frame}

\begin{frame}[fragile]
\frametitle{The fastest compilation course on Earth (2)}

The most common way to produce a \textbf{program} is by using a
\textbf{compiler}.

\vspace{0.3cm}

A \textbf{compiler} takes as input a \textbf{source file} and produces
a \textbf{program}.

\vspace{0.3cm}

The \textbf{source file} must contain a program expressed using a
particular \textbf{programming language} the compiler can understand.

Each \textbf{programming language} specifies where the execution will
start.

In C/C++, the starting point is the \textbf{main} function. Each C
function produces a symbol. See ex1.


\end{frame}


\begin{frame}[fragile]
\frametitle{The fastest compilation course on Earth (3)}

\textbf{Starting here, we will deal with more precisely with C/C++.}

\vspace{0.3cm}

Soon enough, programmers needed a way to keep their code organized:
the idea of splitting the \textbf{source file} into several ones
emerged.

\vspace{0.3cm}

It was easy enough: each \textbf{source file} is compiled into an
\textbf{object file}. All the \textbf{object files} are then merged
together to produce the final \textbf{program}. This new step is
called the \texttt{link} and is realized by a program called
\texttt{ld}.

\vspace{0.3cm}

There is only one problem: how to handle the dependencies?

\end{frame}

\begin{frame}[fragile]
\frametitle{The fastest compilation course on Earth (4)}

\textbf{aux.c} - \texttt{gcc -c aux.c}
\begin{verbatim}
int mysubroutine ()
{
  // Do something...
}
\end{verbatim}

\textbf{main.c} - \texttt{gcc -c main.c}
\begin{verbatim}
int main ()
{
  mysubroutine (); // THIS DOES NOT EXIST!
}
\end{verbatim}


\texttt{gcc aux.o main.o -o myprogram}

\end{frame}


\begin{frame}[fragile]
\frametitle{The fastest compilation course on Earth (5)}

The solution: allowing the programmer to \textit{promise} that a
particular symbol (function) will be provided later.

\textbf{main.c}
\begin{verbatim}
int mysubroutine ();

int main ()
{
  mysubroutine (); // THIS EXISTS!
}
\end{verbatim}

\end{frame}


\begin{frame}[fragile]
\frametitle{The fastest compilation course on Earth (6)}

Still, this solution does not scale well\ldots

\textbf{aux1.c}
\begin{verbatim}
int mysubroutine1 () { /* something */ }
\end{verbatim}

\textbf{aux2.c}
\begin{verbatim}
int mysubroutine1 ();
int mysubroutine2 () { mysubroutine2(); }
\end{verbatim}

\textbf{aux3.c}
\begin{verbatim}
int mysubroutine1 ();
int mysubroutine2 ();
int mysubroutine3 () { mysubroutine2() + mysubroutine1(); }
\end{verbatim}


What happend if I modify \texttt{aux1.c}? I must modify \textbf{ALL}
the other files as well. The compiler always believe in what I
promise, if I fail to modify the other \textbf{source files}
accordindly, it will fail and tell me that I failed to follow my
promises!

\end{frame}



\begin{frame}[fragile]
\frametitle{The fastest compilation course on Earth (7)}

To solve this issue, the \textbf{pre-processor} has been invented.

\vspace{0.3cm}

A \textbf{pre-processor} is a tool which modifies a \textbf{source
  file} before the compiler to ease the writing of a program.

\vspace{0.3cm}

Basically, it copy, paste and replace piece of texts and has
absolutely no idea of the underlying \textbf{programming language}.

\vspace{0.3cm}

For C/C++, this tool is called \textbf{C pre-processor}
(i.e. \texttt{cpp}).

\vspace{0.3cm}

It is easy to recognize in a C/C++ \textbf{source file} which lines
are \textbf{preprocessing directives}: they all begin with the \#
character:

\begin{itemize}
\item \texttt{\#include}
\item \texttt{\#define}
\item \texttt{\#ifdef} \texttt{\#ifndef}
\item \texttt{\#pragma}
\end{itemize}
\end{frame}


\begin{frame}[fragile]
\frametitle{The fastest compilation course on Earth (8)}

Therefore, \textbf{headers} (i.e. what comes at the beginning) emerged
as a universal \textbf{design pattern}.

\textbf{aux.h}
\begin{verbatim}
int mysubroutine1 ();
int mysubroutine2 ();
int mysubroutine3 ();
\end{verbatim}

\textbf{aux1.c}
\begin{verbatim}
#include "/tmp/aux1.h"
int mysubroutine1 () {}
\end{verbatim}

\textbf{main.c}
\begin{verbatim}
#include "/tmp/aux1.h"
int main { mysubroutine3(); }
\end{verbatim}
\end{frame}


\begin{frame}[fragile]
\frametitle{The fastest compilation course on Earth (9)}

The C/C++ syntax forbids to declare several times the same
function/\textbf{symbol}.

\textbf{aux1.h}
\begin{verbatim}
int mysubroutine1 (){}
\end{verbatim}
\textbf{aux2.h}
\begin{verbatim}
#include "/tmp/aux1.h"
int mysubroutine1 ();
\end{verbatim}
\textbf{aux3.h}
\begin{verbatim}
#include "/tmp/aux1.h"
#include "/tmp/aux2.h" // ERROR HERE
int mysubroutine1 ();
\end{verbatim}

\textbf{main.c}
\begin{verbatim}
#include "/tmp/aux3.h"
int main { mysubroutine3(); }
\end{verbatim}
\end{frame}

\begin{frame}[fragile]
\frametitle{The fastest compilation course on Earth (10)}

\textbf{Headers} are therefore defined as follow:

\begin{verbatim}
#ifndef MYPROJECT_AUX1_HH
# define MYPROJECT_AUX1_HH
int mysubroutine1 () {}
#endif //! MYPROJECT_AUX1_HH
\end{verbatim}

It will trim the header if it has already been included elsewhere.
It also save time as parsing the header can count up to 50\% of the
total compilation time of a complex project.

However, all of this is \textbf{good practices}, no pre-processor nor
compiler will never check that this is done properly\ldots\\ What
happens if someone else defines \texttt{MYPROJECT\_AUX1\_HH}?

\end{frame}



\begin{frame}[fragile]
\frametitle{The fastest compilation course on Earth (11)}

Do we really want that?

\begin{verbatim}
#include "/tmp/aux1.h"
\end{verbatim}

What if I do not have a \texttt{tmp} folder? What if I want to store
the header somewhere else?

\vspace{0.3cm}

Let's change it to:

\begin{verbatim}
#include "aux1.h"
\end{verbatim}

\vspace{0.3cm}

What we win in flexibility, we lose it by having to tune the compiler
flags:

\begin{verbatim}
gcc -I/tmp main.c -o myprogram
\end{verbatim}

\vspace{0.3cm}

But this scales (quite) well:

\begin{verbatim}
gcc -I/tmp -I/myfolder -I/my3rdfolder main.c -o myprogram
\end{verbatim}

\end{frame}

\begin{frame}[fragile]
\frametitle{The fastest compilation course on Earth (12)}

Ok, so \texttt{headers} are nice to share symbols among the same
program! But how do that between multiple programs?

\vspace{0.3cm}

\ldots and the library was invented!

\vspace{0.3cm}

Libraries comes in two flavors:

\begin{itemize}
\item \textbf{static}
\item or \textbf{dynamic}
\end{itemize}

Both kinds of libraries fulfills the same goal: providing \textit{set
  of symbols} with no entry point to factor usefull algorithms.

\end{frame}

\begin{frame}[fragile]
\frametitle{The fastest compilation course on Earth (13)}

\textbf{Static} libraries are archives containing \textbf{object
  files}.

This kind of archives can be manipulated using the \texttt{ar} tool.

\vspace{0.3cm}

\begin{verbatim}
gcc -c aux.c -o aux.o
ar rcs aux.a aux.o

gcc -c main.c -o main.o
gcc main.o aux.a -o myprogram
\end{verbatim}

\vspace{0.3cm}

\begin{itemize}
\item Easy to use
\item Widely supported
\item \ldots but largely replaced by \textbf{dynamic libraries}!
\end{itemize}

\end{frame}



\begin{frame}[fragile]
\frametitle{The fastest compilation course on Earth (14)}

\textbf{Dynamic libraries} is a mechanism which allows to resolve
symbols are runtime.

\begin{verbatim}
gcc -I/tmp -c aux.c -o aux.o
gcc -shared aux.o -o libaux.so

gcc -I/tmp -c main.c -o main.o
gcc main.o -L. -laux -o myprogram
\end{verbatim}

All shared libraries should be called \textbf{libSOMETHING.so} so that
one can link against it using the \textbf{-lSOMETHING} flag.
\end{frame}


\begin{frame}[fragile]
\frametitle{The fastest compilation course on Earth (15)}

\textbf{Dynamic libraries} works as follow:

\begin{itemize}
\item At link time, the flag \texttt{-L} is used to retrieve the
  required dynamic libraries linked with \texttt{-l}. This is only a
  security to make sure that all the symbols declared in the headers
  really exists. However, \textit{no compiled code} is included at
  this time!
\item At runtime, the dynamic libraries are loaded before the
  execution starts. The lookup is different from the compile time (in
  particular the \texttt{-L} flag do no impact this step).
\end{itemize}

The lookup is done as follow:
\begin{itemize}
  \item \texttt{LD\_LIBRARY\_PATH} environment variable
  \item rpath: additional run-time path specified during the compilation
  \item \texttt{/etc/ld.so.cache}: cache file containing system libraries
  \item default hard-coded paths: /lib and /usr/lib.
\end{itemize}

This is a simplified process, each step has exceptions which are not
mentioned here. It is also highly system dependent, this one is for
the \texttt{ld.so} Linux linker.
\end{frame}

\begin{frame}[fragile]
\frametitle{The fastest compilation course on Earth (16)}

Ok, so let's wrap it up. To compile a software, one needs to setup the
compiler and give it the following information:

\begin{description}
\item[\textbf{Header} location] using the \texttt{-I} flag,
\item[\textbf{Dynamic libraries} location] using the \texttt{-L} flag,
\item[\textbf{Dynamic libraries} dependencies] using the \texttt{-l}
  flag.
\end{description}

Additional flags are also useful:
\begin{description}
\item[Warning management] using the \texttt{-W*} flag.\\ I really highly
  encourage you to use \texttt{-Werror -Wall} in all your projects!
\item[Debug] using the \texttt{-g} and \texttt{-ggdbN} flags,
\item[Optimization] using the \texttt{-On} flags.
\end{description}

Also, supporting additional compilers are useful for cross-compiling
(some robots are not using x86 architecture) or obtaining improved
performances (such as \texttt{icc} the Intel Compiler).
\end{frame}


\begin{frame}[fragile]
\frametitle{Build-chain tools}

So now you want to have a properly designed package, you must make it
configurable to change the flags and also retrieve properly the
compilation flags needed to build your software.

\vspace{0.3cm}

The classic way is writing a \texttt{Makefile}. This file teaches the
\texttt{make} program how to generate certain program.

\begin{verbatim}
# Create the toto.txt file.
make toto.txt
\end{verbatim}

Its syntax is basically:
\begin{verbatim}
output: input1 input2
    command which generates output from input1/input2 (shell)
\end{verbatim}
\end{frame}

\texttt{output} is called a \textbf{target}.

\begin{frame}[fragile]
\frametitle{Build-chain tools}

This approach has several drawbacks:

\begin{itemize}
\item \texttt{Makefile} are not configurable,
\item \texttt{Makefile} are not human friendly,
\item it is difficult to write portable and reusable \texttt{Makefile}s.
\end{itemize}

Therefore, we will let CMake generate \texttt{Makefile}s for us!

\end{frame}

\begin{frame}[fragile]
\frametitle{Depending on other libraries}

It is extremely difficult to guess third party dependencies flags, but
one tool can help us: \texttt{pkg-config}.

\vspace{0.2cm}

\texttt{pkg-config} can display what flags are required for a
particular package.

\begin{verbatim}
pkg-config --cflags my-package
pkg-config --libs my-package
\end{verbatim}

The only condition is that \texttt{my-package} provides a \texttt{.pc}
file containing the required information.
\end{frame}



\begin{frame}[fragile]
\frametitle{Installing data}

A package is more than a set of programs, it contains documentation,
tests\ldots So one last question remains: how to structure your
package and where to install the data?

\vspace{0.3cm}

On Linux, the folder hierarchy is defined by the \textit{Filesystem
  Hierarchy Standard} aka FHS.

\begin{description}
\item[\texttt{bin}] binaries
\item[\texttt{lib}] static and dynamic libraries
\item[\texttt{include}] headers
\item[\texttt{share}] platform-independent files
  \begin{description}
  \item[\texttt{doc}] documentation
  \end{description}
\end{description}

Several standard prefixes exist: \texttt{/}, \texttt{/usr},
\texttt{/usr/local} and additional ones are often defined (on your
home account for RobotPkg).

\end{frame}

\begin{frame}[fragile]
\frametitle{Package structure}

Before being installed, the structure is roughly the same:

\begin{description}
\item[\texttt{doc}] documentation (should be \texttt{share/doc})
\item[\texttt{include}] headers
\item[\texttt{share}] platform-independent files
\item[\texttt{src}] source files and private headers
\item[\texttt{tests}] tests
\item[\texttt{README}, \texttt{NEWS}, \texttt{LICENSE}] root-level
  documentation.
\end{description}
\end{frame}


\begin{frame}[fragile]
\frametitle{Using CMake}

Compiling a software using CMake is very easy:

\begin{verbatim}
cd my-project-0.1
mkdir build
cd build
ccmake .. # cmake .. to avoid using the GUI
make
make install # for instance...
\end{verbatim}

\end{frame}


\begin{frame}[fragile]
\frametitle{CMake primer}

CMake is a tool which will handle the build-chain by generating
appropriate \texttt{Makefile}s semi-automatically.

\vspace{0.3cm}

To do so, it parses \texttt{CMakeLists.txt} files which contains
building instructions. This is a pseudo-script language with limited
features but which is quite easy to understand.

\vspace{0.3cm}

API changes depending on the CMake version, so be extremely careful!
Last stable version is 2.8, it is reasonale to be backward dependent
up-to 2.6.

\end{frame}


\begin{frame}[fragile]
\frametitle{CMake syntax}

The language is case independent.

\begin{verbatim}
# This is a comment....

# Variable assignment (A=1)
SET(A 1)

# This set MYLENGTH to 9.
STRING(LENGTH "my string" MYLENGTH)

MACRO FOO()
 MESSAGE("This is the function called FOO")
ENDMACRO()
\end{verbatim}
\end{frame}

\begin{frame}[fragile]
\frametitle{CMake API overview}

You can read the documentation online or by typing \texttt{man cmake}:
\url{http://www.cmake.org/cmake/help/cmake2.6docs.html}

\begin{description}
\item[\texttt{ADD\_EXECUTABLE}] compiles a program.
\item[\texttt{ADD\_LIBRARY}] compiles a static or dynamic library.
\item[\texttt{ADD\_SUBDIRECTORY(SUBDIR)}] includes
\item[\texttt{ADD\_TEST}] compiles a test (i.e. a program which is
  executed when \texttt{make test} is launched).
\item[\texttt{INSTALL}] install a file or a target in the installation
  prefix.
\end{description}
\end{frame}

\begin{frame}[fragile]
\frametitle{CMake API overview (2)}

\begin{description}
\item[\texttt{PROJECT}] Define the project name (and set the
  \texttt{PROJECT\_NAME} variable).
\item[\texttt{TARGET\_LINK\_LIBRARIES}] Link a target against a
  library.
\item[\texttt{ENABLE\_TESTING}] Generate a \texttt{test} target.
\item[\texttt{INCLUDE\_DIRECTORIES}] Add a new directory in which
  headers will be searched (i.e. \texttt{-I} GCC compilation flag).
\item[\texttt{LINK\_DIRECTORIES}] Add a new directory in which
  libraries will be searched (i.e. \texttt{-L} GCC compilation flag).
\item[\texttt{ADD\_DEFINITIONS}] Add a new pre-processor definition
  (i.e. \texttt{-D} GCC compilation flag).
\item[\texttt{SET\_TARGET\_PROPERTIES}] Set compilation flags manually
  for a specific target.
\end{description}
\end{frame}

\begin{frame}[fragile]
\frametitle{CMake API overview (3)}

Additional, general purpose, macros are: \texttt{STRING},
\texttt{MATH}.

\vspace{0.3cm}

Logic statements are supported: \texttt{IF}, \texttt{FOREACH},
\texttt{WHILE}\ldots but it is usually a bad idea to write too complex
CMake scripts.

\vspace{0.3cm}

One can also look for files (\texttt{FIND\_FILE}), libraries
(\texttt{FIND\_LIBRARY}), packages (\texttt{FIND\_PACKAGES}), path
(\texttt{FIND\_PATH}).

\vspace{0.3cm}

Macros for Boost, BLAS and numerous other libraries are available. Do
not trust them too much as most of them are buggy (really!).

\end{frame}


\begin{frame}[fragile]
\frametitle{Configuring files}

Sometimes, it is necessary to generate files by hand. For example:
\begin{itemize}
\item Generate a header with the project version number.
\item Generate a \texttt{.pc} file from a template file.
\end{itemize}

\texttt{CONFIGURE\_FILE} can be used to do so. It will look for the
\texttt{@XYZ@} patterns and replace them by the value of the variable
\texttt{XYZ} in CMake.

\end{frame}

\begin{frame}[fragile]
\frametitle{CMake cache}

A (not so) food feature of CMake is the cache. It allows some values
to be remembered between two \texttt{Makefile} generation to gain time
and remember the user choices.


See \texttt{CMakeCache.txt} file in the build directory.

\vspace{0.3cm}

Additional options of the \texttt{SET} macro can be used to create a
cached variable. One can set the cached variable type, description in
\texttt{ccmake}, etc. See also \texttt{OPTION}.

\end{frame}


\begin{frame}[fragile]
\frametitle{CMake tutorial}

A package is available here:

\vspace{0.1cm}

\texttt{git clone \url{git://github.com/thomas-moulard/cmake-tutorial.git}}

\vspace{0.3cm}

It contains two libraries, one program, one test. The goal is to write
the corresponding \texttt{CMakeLists.txt} using CMake.

\begin{itemize}
\item Compile the shared libraries \texttt{libfoo.so} and \texttt{libbar.so}
\item Compile the program \texttt{my-program}
\item Compile the test \texttt{bar-test} using \texttt{bar.cc}
\item Write the installation rules.
\item Bonus: generate \texttt{foo.pc} and \texttt{bar.pc} from a
  template file called \texttt{pkg-config.pc.in} and install them in
  \texttt{<prefix>/lib/pkgconfig}.
\end{itemize}

\end{frame}


\begin{frame}[fragile]
\frametitle{One last thing\ldots}

CMake is missing some crucial elements such as proper
\texttt{pkg-config} and \texttt{doxygen} support. To avoid reinventing
the wheel, the CMake macros used by GEPETTO are available on GitHub:

\vspace{0.2cm}

\url{https://github.com/jrl-umi3218/jrl-cmakemodules}

\vspace{0.3cm}

In your favorite Git enabled project, type:\\
\texttt{git submodule add \url{git://github.com/jrl-umi3218/jrl-cmakemodules.git} cmake}

\end{frame}


\end{document}
